\documentclass[11pt]{article} % Font size - 10pt, 11pt or 12pt

\usepackage[hmargin=1.25cm, vmargin=1.0cm]{geometry} % Document margins

\usepackage{marvosym} % Required for symbols in the colored box
\usepackage{ifsym} % Required for symbols in the colored box
\usepackage{scrextend}

\setlength{\parindent}{0pt}


\usepackage[usenames,dvipsnames]{xcolor} % Allows the definition of hex colors

% Fonts and tweaks for XeLaTeX
\usepackage{fontspec,xltxtra,xunicode}
\defaultfontfeatures{Mapping=tex-text}
%\setromanfont[Mapping=tex-text]{Hoefler Text} % Main document font
%\setsansfont[Scale=MatchLowercase,Mapping=tex-text]{Gill Sans} % Font for your name at the top
%\setmonofont[Scale=MatchLowercase]{Andale Mono}

% Colors for links, text and headings
\usepackage{hyperref}
\definecolor{linkcolor}{HTML}{506266} % Blue-gray color for links
\definecolor{shade}{HTML}{F5DD9D} % Peach color for the contact information box
\definecolor{text1}{HTML}{2b2b2b} % Main document font color, off-black
\definecolor{headings}{HTML}{701112} % Dark red color for headings
% Other color palettes: shade=B9D7D9 and linkcolor=A40000; shade=D4D7FE and linkcolor=FF0080

\hypersetup{colorlinks,breaklinks, urlcolor=linkcolor, linkcolor=linkcolor} % Set up links and colors

\usepackage{fancyhdr}
\pagestyle{fancy}
\fancyhf{}
% Headers and footers can be added with the \lhead{} \rhead{} \lfoot{} \rfoot{} commands
% Example footer:
%\rfoot{\color{headings} {\sffamily Last update: \today}. Typeset with Xe\LaTeX}

\renewcommand{\headrulewidth}{0pt} % Get rid of the default rule in the header

\usepackage{titlesec} % Allows creating custom \section's

% Format of the section titles
\titleformat{\section}{\color{headings}
\scshape\Large\raggedright}{}{0em}{}[\color{black}\titlerule]

\titlespacing{\section}{0pt}{0pt}{5pt} % Spacing around titles

\begin{document}

\color{text1} % Sets the default text color for the whole document to the color defined as 'text1'


\begin{minipage}[t]{0.49\textwidth}
\vspace{0pt}
{\huge{Salvatore Cardamone}}\\ % Your name
{\color{headings} \Large{Curriculum Vitae}}
\end{minipage}\hfill
\begin{minipage}[t]{0.49\textwidth}
\vspace{1pt}
\colorbox{shade}{\textcolor{text1}{
\begin{tabular}{c|p{7cm}}
\raisebox{-3pt}{\textifsymbol{18}} & 7 Candover St., London, W1W 7DN \\ % Address
\raisebox{-2pt}{\Mobilefone} & +44 (0)7794 798 945 \\ % Phone number
\raisebox{-1pt}{\Letter} & \href{mailto:sc2018@cam.ac.uk}{sav.cardamone@gmail.com} \\ % Email address
\end{tabular}
}}\\[10pt]
\end{minipage}\\




\section{Employment}
\begin{tabular}{rl}
2021 -- Present & \textbf{Senior Signal Processing Engineer, CoMind Technologies}\\
&I have led the internal development of the signal processing pipeline required for the\\
&procurement of neurological signals from experimental data. I have led work to scope\\
&out a comprehensive set of work packages that will allow for this processing to be\\
&conducted in a quasi-realtime fashion, thereby allowing the experimentalists to optimise\\
&configurations on-the-fly. I am also involved in the development of embedded and low-level\\
&software, including control suites for digital acquisition and waveform generation.\\
&\\
2018 -- 2021 & \textbf{Senior Engineer, Cambridge Consultants Ltd}\\
&My work largely dealt with the physical layer in 5G, projects including:\\
&\hspace{5mm}*\hspace{3mm}Porting of performance critical sections of PXSCH to CUDA.\\
&\hspace{5mm}*\hspace{3mm}Development of in-house implementation of the full NR PHY (primarily FEC\\
&\hspace{5mm}\hphantom{*}\hspace{3mm}and channel estimation/ equalisation) for an embedded platform.\\
&\hspace{5mm}*\hspace{3mm}Benchmarking throughput of LDPC codec on a CEVA DSP.\\
&I worked on other wireless standards (BR/EDR and BLE) for low-power embedded\\ 
&systems. I was involved in the winning and subsequent completion of work\\ 
&commissioned by a multinational hardware vendor, concerning the appraisal of a novel\\ 
&SIMD HPC platform for specific workloads in geophysical resource exploration.\\
&I led the quantum technologies special interest group, obtaining internal funding\\
&for the development of several internal demonstrators.\\ 
&\\
2016 -- 2018 & \textbf{Postdoctoral Research Associate, The University of Cambridge}\\
& Postdoctoral research associate with Dr Alex Thom, working on the acceleration of\\ 
&common HPC workloads in computational chemistry with dataflow FPGAs. This involved\\ 
&the design of benchmarks using conventional multithreaded/ multicore systems and GPUs,\\ 
&and optimisation of algorithms/ memory-access patterns for each platform. I was\\ 
&involved in discussions with Maxeler Technologies, giving feedback on useful features\\ 
&that could be incorporated into their high-level synthesis toolchain to facilitate the\\ 
&porting of academic HPC codebases to their platform.\\
\end{tabular}
\\

\section{Education} 

\begin{tabular}{rl} % Start a table with two columns, one for dates and one for qualifications

2012 -- 2016 & \textbf{Doctor of Philosophy} \\ 
& \textsc{Theoretical Chemistry} \textit{The University of Manchester}\\
&Researched the use of machine learning in computational chemistry to aid in the prediction\\ 
&of atomic multipole moments, obtained through Bader decomposition, as a function of molecular\\
&conformation. I worked with novel domain of application metrics to quantify the validity\\
&of training sets on-the-fly, and consequent iterative training set refinement strategies.\\
&\\

2009 -- 2012 & \textbf{Bachelor of Science}, 1\textsuperscript{st} Class (Hons)\\ 
& \textsc{Biochemistry}, \textit{The University of Sheffield}\\

\end{tabular}\\[10pt]

\section{Computer Skills} 

\begin{tabular}{rl}
\textbf{Fluent}
&C, C++17, bash, \textsc{FORTRAN90}, MATLAB, python, git, OpenCL,\\ 
&CUDA, MPI, OpenMP\\ 
\textbf{Competent}
& perl, \LaTeX, java, Haskell, VHDL\\
\end{tabular}
\\

\newpage

\section{Awards and Prizes} 
\begin{tabular}{rl}

\textsc{July 2015} & \large{\textbf{BBSRC Funded Studentship}}\\
& \textit{Daresbury National Laboratory, Warrington}\\
& Implemented a novel isokinetic ensemble thermostatting methodology within the\\ &framework of the DL\_POLY molecular dynamics package.\\
\\

\textsc{Summer 2011} & \large{\textbf{The Biochemical Society Bursary Award}}\\
& \textit{Waltho Lab, University of Sheffield}\\
& Performed molecular dynamics simulations on phosphoglucokinase.\\
\\

\textsc{Summer 2007} & \large{\textbf{The Nuffield Foundation Scholarship}}\\
& \textit{Loadman Lab, Bradford Institute for Cancer Therapeutics}\\
& Evaluated the efficacy of a tumour-specific compound targeting matrix metalloproteinases.\\
\\

\end{tabular}

\section{Conferences}

\begin{tabular}{rl}
	\textsc{talks} 
    & \textbf{FPL, 2018}, \textit{Dublin, Ireland}\\
    & FPGAs and Quantum Monte Carlo: Automated Porting using CAOS\\
    & \textbf{Paderborn Centre for Parallel Computing, 2018}, \textit{Paderborn, Germany}\\
    & FPGAs in a Multiprocessing Environment for Quantum Monte Carlo\\
	& \textbf{Xilinx, 2018}, \textit{Dublin, Ireland}\\
    & Numerical Precision in Quantum Monte Carlo and Importance for FPGAs\\
    & \textbf{HiPEAC, 2017}, \textit{St\"{o}ckholm, Sweden}\\
    & FPGA Acceleration of Diffusion Monte Carlo\\
    & \textbf{Symposium on Computational Chemistry, 2015}, \textit{Manchester, UK}\\
    & Conformational Sampling of Vibrational Modes for Machine Learning\\
    \\
    \textsc{posters} & \textbf{ACS National Meeting 2016}, \textit{San Diego, USA}\\
    & A Novel Domain of Application for Machine Learning\\
    & \textbf{Quantum Monte Carlo Conference}, \textit{Tuscany, Italy}\\
    & An Analysis of Bader Decompositions of Molecular Systems at Various Levels of Theory\\
\end{tabular}\\

\section{Teaching Experience}

\textbf{University of Cambridge}\\
Academic supervisor for the third year courses:
\begin{itemize}
	\item Theoretical Techniques (A4)
    \item Perturbation Theory and Further Quantum Mechanics (B7)
    \item Electronic Structure Theory (C6)
\end{itemize}
Have supervised a summer student and an MPhil candidate in our lab, both completing successful projects in quantum chemistry/ software development. Also a senior laboratory demonstrator for computational and theoretical laboratories in chemistry part of natural sciences tripos.\\
\\
\textbf{University of Manchester}\\
Supervised two successful Masters students over the course of my PhD, one of which resulted in an academic publication.\\


\section{Extracurricular Activities} 

Along with my dance partner, I was the national pre-amateur champion in ballroom dancesport. We competed internationally in China and mainland Europe. I was the president of the ballroom and latin dancesport societies at both the Universities of Manchester and Sheffield. I am a fluent speaker of Italian and am competent in French.

\newpage

\begin{minipage}[t]{0.49\textwidth}
\vspace{0pt}
{\huge{Salvatore Cardamone}}\\ % Your name
{\color{headings} \Large{Academic Publications}}
\end{minipage}\hfill
\begin{minipage}[t]{0.49\textwidth}
\vspace{2cm}
%\colorbox{shade}{\textcolor{text1}{
%\begin{tabular}{c|p{7cm}}
%\raisebox{-3pt}{\textifsymbol{18}} & 54 Howard Road, Cambridge, CB5 8QP \\ % Address
%\raisebox{-2pt}{\Mobilefone} & +44 (0)7794 798 945 \\ % Phone number
%\raisebox{-1pt}{\Letter} & \href{mailto:sc2018@cam.ac.uk}{sc2018@cam.ac.uk} \\ % Email address
%\end{tabular}
%}}\\[10pt]
\end{minipage}\\
\section{Publications to Date} 

Those publications for which I have conducted the majority of work have been labeled with an asterisk.\\

Peverelli, F, Rabozzi, M, Cardamone, S, Del Sozzo, E, Thom, AJWT, Santambrogio, MD \& Di Tucci, L, Automated Acceleration of Dataflow-Oriented C Applications on FPGA-Based Systems, \textit{IEEE 27\textsuperscript{th} Annual International Symposium on Field-Programmable Custom Computing Machines (FCCM)}, \textbf{2019}\\

*Cardamone S, Kimmitt JRR, Burton HGA, Todman TJ, Li S, Luk, W \& Thom AJWT, Field-Programmable Gate Arrays and Quantum Monte Carlo: Power Efficient Co-processing for Scalable High-Performance Computing, \textit{International Journal of Quantum Chemistry}, \textbf{2019}, 119(12), 12817 \\

Jensen K, Benson R, Cardamone S \& Thom AJWT, Simple and Fast Quasidiabats from Self-Consistent Field Metadynamics, \textit{Journal of Chemical Theory and Computation}, , \textbf{2018}, 14(9), 4269-4639 \\

Zielinski F et al., Geometry Optimization with Machine Trained Topological Atoms, \textit{Nature Scientific Reports}, \textbf{2017}, 7(1), 12817 \\

*Cardamone S, Caine B, Blanch E \& Popelier PLA, The Computational Prediction of Raman and ROA Spectra of Charged Histidine Tautomers in Aqueous Solution, \textit{Physical Chemistry Chemical Physics}, \textbf{2016}, 18(39), 27377-27389 \\

Maxwell P, Cardamone S \& Popelier PLA, The Prediction of Topologically Partitioned Intra-Atomic and Inter-Atomic Energies by the Machine Learning Method Kriging, \textit{Theoretical Chemistry Accounts}, \textbf{2016}, 135(8), 195-210\\

*Cardamone S \& Popelier PLA, Prediction of Conformationally Dependent Atomic Multipole Moments in Carbohydrates, \textit{Journal of Computational Chemistry}, \textbf{2015}, 36(32), 2361-2373\\

*Hughes T, Cardamone S \& Popelier PLA, Realistic Sampling of Amino Acid Geometries for a Multipolar Polarizable Force Field, \textit{Journal of Computational Chemistry}, \textbf{2015}, 36(24), 1844-1857\\

*Cardamone S, Hughes T \& Popelier PLA, Multipolar Electrostatics, \textit{Physical Chemistry Chemical Physics}, \textbf{2014}, 16(22), 10367-10387\\

\end{document}  
